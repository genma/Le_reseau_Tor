\documentclass{beamer}
\mode<presentation> {
\usepackage{color}
\definecolor{bottomcolour}{rgb}{0.21,0.11,0.21}
\definecolor{middlecolour}{rgb}{0.21,0.11,0.21}
\setbeamercolor{structure}{fg=white}
\setbeamertemplate{frametitle}[default]%[center]
\setbeamercolor{normal text}{bg=black, fg=white}
\setbeamertemplate{background canvas}[vertical shading]
[bottom=bottomcolour, middle=middlecolour, top=black]
\setbeamertemplate{items}[circle]
\setbeamertemplate{navigation symbols}{} %no nav symbols
\setbeamercolor{block title}{use=structure,fg=white,bg=structure.fg!50!red!50!blue!100!green}
\setbeamercolor{block body}{parent=normal text,use=block title,bg=block title.bg!5!white!10!bg,fg=white}
\setbeamertemplate{navigation symbols}{}
%-- Ajout du nb de pages en bas de page %
\addtobeamertemplate{footline}{\hfill\insertframenumber/\inserttotalframenumber\hspace{2em}\null}
}
\usepackage{graphicx} 
\usepackage{booktabs} 
\usepackage[utf8]{inputenc}  
\usepackage[T1]{fontenc}  
\usepackage{geometry}     
\usepackage[francais]{babel} 
\usepackage{eurosym}
\usepackage{verbatim}
\usepackage{ragged2e}
\justifying
\input{cc_beamer}
\title[Tor et le Tor Browser Bundle]{Le réseau Tor} 
\author{Genma}
\begin{document}
\begin{frame}
	\titlepage
	\begin{center}
		\includegraphics[scale=0.2]{./images/logo_tor.jpg}
		\\		
		\CcGroupByNcSa{0.83}{0.95ex}\\[2.5ex]
		{\tiny\CcNote{\CcLongnameByNcSa}}
		\vspace*{-2.5ex}
	\end{center}
\end{frame}

\begin{frame}
\frametitle{\includegraphics[scale=0.4]{./images/Genma.jpg} \ \ \  A propos de moi  }
\begin{columns}[c] 

\column{.55\textwidth} 
\textbf{Où me trouver sur Internet?}
\begin{itemize}
\item Le Blog de Genma
\item Twitter : @genma
\end{itemize}
\textbf{Mes projets - contributions}
\\ Plein de choses dont:
\begin{itemize}
\item Promotion de Tor, traduction...
\item A.I.\up{2} Apprenons l'Informatique, Apprenons Internet
\end{itemize}

\column{.5\textwidth} 
\includegraphics[width=5cm,height=5cm]{./images/blog.png} 
\end{columns}
\end{frame}


%----------------------------------------------------------------------------------------
\begin{frame}
\frametitle{Objectif de la présentation}

\justifying{
La présentation a une durée d'une heure, questions comprises.
Le but est de vous aider à comprendre ce qu'est Tor, son principe, ses usages. Tous les aspects de Tor ne pourront bien évidememment pas être vus ;-) \\
$\Rightarrow$ Nous pourrons poursuivre la discussion sur le stand, dans la limite de mes propres connaissances.
\\~\\
}

\begin{block}{ATTENTION}
\justifying{Bien qu'un long travail de préparation ait été fait, des erreurs, des simplifications peuvent figurées dans cette présentation ou dans les propos tenus. C'est la documentation du projet \\ \url{https://www.torproject.org} qui fait foi. 
}
\end{block}
\end{frame}


%----------------------------------------------------------------------------------------
\begin{frame}
\begin{center}
\Huge{Introduction}
\\~\\ \includegraphics[scale=0.4]{./images/logo_tor.jpg}
\end{center}
\end{frame}
%----------------------------------------------------------------------------------------
\begin{frame}
\frametitle{Présentation du réseau Tor}

\begin{block}{C'est quoi Tor?}
\justifying{
Tor est l'acronyme de Tor onion routing, littéralement « le routeur oignon Tor », en référence au \emph{routage en oignon}.
\begin{itemize}
\justifying{
\item Le réseau Tor est composé de routeurs organisés en couches, appelés nœuds de l’oignon, qui transmettent de manière anonyme des flux TCP.
\item Tor est un logiciel libre (logiciel client et serveur) soutenu par l'organisation The Tor Project.
}
\end{itemize}
}
\end{block}
\end{frame}
%----------------------------------------------------------------------------------------
\begin{frame}
\begin{center}
\Huge{A quoi sert TOR?}
\\~\\ \includegraphics[scale=0.4]{./images/logo_tor.jpg}
\end{center}
\end{frame}
%----------------------------------------------------------------------------------------
\begin{frame}
\frametitle{A quoi sert TOR?}

\begin{block}{Ce que l'usage de Tor permet de faire}
\justifying{
\begin{itemize}
\justifying{
\item  d'échapper au fichage publicitaire,
\item  de publier des informations sous un pseudonyme,
\item  d'accéder à des informations en laissant moins de traces,
\item  de déjouer des dispositifs de filtrage (sur le réseau de son entreprise, de sa Université, en Chine ou en France…),
\item  de communiquer en déjouant des dispositifs de surveillances,
\item  de tester son pare-feu,
\item  … et sûrement encore d'autres choses.
}
\end{itemize}
$\Rightarrow$ Tor dispose également d'un système de « services cachés » qui permet de fournir un service en cachant l'emplacement du serveur.
}
\end{block}
\end{frame}

%----------------------------------------------------------------------------------------
\begin{frame}
\frametitle{Qu'est-ce que TOR?}

\begin{block}{En résumé}
\justifying{
Tor peut donc être vu comme un proxy socks5 un peu particulier, par lequel on peut faire passer tout type de connexion (dès lors qu'elle est \emph{de type TCP}).
\\~\\
Rq : comme il s'agit d'un proxy, on peut lui associer un autre logiciel qu'un navigateur dès lors que ce dernier accepte de se connecter via un proxy (client de messagerie par exemple).
}
\end{block}
\end{frame}

%----------------------------------------------------------------------------------------
\begin{frame}
\begin{center}
\Huge{Comment fonctionne Tor ? }
\\~\\ \includegraphics[scale=0.4]{./images/logo_tor.jpg}
\end{center}
\end{frame}
%----------------------------------------------------------------------------------------
\begin{frame}
\frametitle{Comment fonctionne Tor ?}
\begin{center}
\includegraphics[keepaspectratio,width=\textwidth, height=.8\textheight]{images/tor-safe-selection}
\end{center}
\end{frame}
%----------------------------------------------------------------------------------------
\begin{frame}
\frametitle{Comment fonctionne Tor ?}

\begin{block}{Tor - The Onion Router}
\justifying{
\begin{itemize}
\justifying{
\item Tor fait un routage en oignion avec des couches de chiffrement empilées.
\item Le client sélectionne trois noeuds (à minima) parmi les différents "relais" Tor disponibles.
\item Il sélectionne un noeud d'entrée, un noeud de sortie et un ou plusieurs noeuds intermédiaires (selon sa configuration).
}
\end{itemize}
$\Rightarrow$ Attention : Tor ne chiffre pas après le noeud de sortie. Il faut donc utiliser une connexion httpS (les clés étant négociées entre le client et le serveur web final).
}
\end{block}
\end{frame}
%----------------------------------------------------------------------------------------
\begin{frame}
\frametitle{Comment fonctionne Tor ?}
\begin{center}
\includegraphics[keepaspectratio,width=\textwidth, height=.8\textheight]{images/tor-safe-path}
\end{center}
\end{frame}
%----------------------------------------------------------------------------------------
\begin{frame}
\frametitle{Comment fonctionne Tor ?}
\justifying{
Rq : cette partie est peu documenté et assez complexe.
\begin{block}{Echanges de clés}
\justifying{
Après sélection des différents "noeuds-relais" dans le réseau, le client effectue un échange de clés de sessions (Diffie-Hellman).  Il y a une première clé de chiffrement pour le nœud d'entrée, une second clé pour le nœud du milieu et une dernière pour le nœud de sortie.
}
\end{block}
\begin{block}{Une fois le paquet arrivé au noeud de sortie}
\begin{itemize}
\justifying{
\item Chaque noeud envoit donc une clé de session. 
\item Le premier noeud envoie sa clé au client.
\item Le deuxième noeud envoie sa clé au 1er noeud qui la chiffre avec la clé négocié avec le client et l'envoie au client.
\item Idem pour le troisième noeud (qui envoit de sa clé au 2nd noeud...)
}
\end{itemize}
\end{block}
}
\end{frame}

%----------------------------------------------------------------------------------------
\begin{frame}
\frametitle{Comment fonctionne Tor ?}
\begin{center}
\includegraphics[keepaspectratio,width=\textwidth, height=.8\textheight]{images/tor-keys1}
\end{center}
\end{frame}

%----------------------------------------------------------------------------------------
\begin{frame}
\frametitle{Comment fonctionne Tor ?}

\begin{block}{Une fois les clés de sessions échangées,}
\begin{itemize}
\justifying{
\item Le client chiffre le paquet avec les 3 clés de session, l'envoie au 1er noeud qui pèle la première couche, 
\item qui l'envoie au deuxième noeud qui pèle la deuxième couche de chiffrement,
\item et ainsi de suite jusqu'au noeud de sortie.
}
\end{itemize}
\end{block}

\begin{block}{Une fois le paquet arrivé au noeud de sortie}
\begin{itemize}
\justifying{
\item La requête est envoyé au serveur web par le noeud de sortie (qui expose donc son adresse IP).
\item Le noeud de sortie reçoit la réponse.
}
\end{itemize}
\end{block}

\end{frame}

%----------------------------------------------------------------------------------------
\begin{frame}
\frametitle{Comment fonctionne Tor ?}

\begin{block}{Le chemin de retour}
\justifying{
Une fois la réponse reçue,  les différents noeuds savent quel paquet doit suivre quel circruit car ils enregistrent la correspondance IP/Port source (\emph{une sorte de NAT}).
\begin{itemize}
\justifying{
\item Pour le chemin inverse, le dernier noeud chiffre avec sa clé de session ; 
\item le passe au noeud d'avant qui chiffre aussi avec sa clé de session 
\item jusqu'à remonter au client qui déchiffre les couches avec les clés qu'il a reçu lors de la construction du circuit.
}
\end{itemize}
Ainsi au niveau des échanges client noeud d'entrée, le paquet est chiffré par n fois (n étant le nombre de noeud).
}
\end{block}
\end{frame}

%----------------------------------------------------------------------------------------
\begin{frame}
\begin{center}
\Huge{Tor hidden service \\ les services cachés de TOR}
\\~\\ \includegraphics[scale=0.4]{./images/logo_tor.jpg}
\end{center}
\end{frame}
%----------------------------------------------------------------------------------------
\begin{frame}
\frametitle{Tor hidden service - les services cachés de TOR 1/4}
\begin{block}{Caractéristiques des services cachés}
\justifying{
Tor permet aux clients et aux relais d’offrir des services cachés. Il est ainsi possible de proposer l'accès à un serveur web, un serveur SSH, etc, sans révéler son adresse IP aux utilisateurs.
\begin{itemize}
\item Ils portent une adresse qui se termine par .onion.
\item Ces \emph{sites} peuvent n'être accessibles \emph{que} via le réseau Tor.
\item Des wikis et moteurs de recherches référencient ces services cachés.
\end{itemize}
}
\end{block}
\end{frame}

%----------------------------------------------------------------------------------------
\begin{frame}
\frametitle{Tor hidden service - les services cachés de TOR 2/4}

\begin{block}{Fonctionnement}
On a un \emph{double routage en oignion}.
\begin{itemize}
\justifying{
\item 3 relais depuis le client
\item 3 relais depuis le serveur
}
\end{itemize}
\justifying{
$\Rightarrow$ Le client et le serveur se \emph{donne rendez-vous} sur un noeud Tor.
$\Rightarrow$ Le client ne sait rien du serveur et inversement.
}
\end{block}
\end{frame}


%----------------------------------------------------------------------------------------
\begin{frame}
\frametitle{Tor hidden service - les services cachés de TOR 3/4}

\begin{block}{Exemple de sites existants ayant une adresse .oinion}
\justifying{
\begin{itemize}
\item Duckduckgo \url{http://3g2upl4pq6kufc4m.onion}
\item Facebook : \url{https://facebookcorewwwi.onion}
\item Le blog de Stéphane Borztmeyer \\ \url{http://7j3ncmar4jm2r3e7.onion}
\item Techn0polis  d'Amaelle Guiton \\ \url{http://ozawuyxtechnopol.onion}

\end{itemize}
$\Rightarrow$ Il existe des annuaires - wiki listant les sites en .oinion.
}
\end{block}
\end{frame}

%----------------------------------------------------------------------------------------
\begin{frame}
\frametitle{Tor hidden service - les services cachés de TOR 4/4}

\begin{block}{Tutoriaux pour mettre en place un .oinion}
\begin{itemize}
 \justifying{
\item \emph{Configuring Hidden Services for Tor}
\\ \url{https://www.torproject.org/docs/tor-hidden-service.html.en}
\item \emph{Tor, les .onion, le "darknet" à votre portée} par Benjamin Sonntag  \url{https://benjamin.sonntag.fr}
\item \emph{Mon blog dans les oignons} par Stéphane Bortzmeyer
\\ \url{http://www.bortzmeyer.org/blog-tor-onion.html}
}
\end{itemize}
\end{block}
\end{frame}

%----------------------------------------------------------------------------------------
\begin{frame}
\frametitle{Tor hidden service -Tor2Web 1/4}

\begin{block}{Tor2Web}
 \justifying{
Tor2web est un projet qui permet aux utilisateurs d'accéder aux Services Cachés / Hidden Services / adresses en .Onion sans utiliser le Tor Browser. 
\\
Ce projet a été initié par Aaron Swartz and Virgil Griffith. Site officiel \url{https://tor2web.org}
}
\end{block}
\end{frame}


%----------------------------------------------------------------------------------------
\begin{frame}
\frametitle{Tor hidden service - Tor2Web 2/4}

\begin{block}{Proxy vers les hidden services}
 \justifying{
Tor2Web agit comme un proxy spécialisé ou une sorte d'intermédiaire entre les services cachés et les utilisateurs en rendant visible et accessibles ces services aux personnes qui sont pas connectés à Tor.
}
\end{block}

\begin{block}{Utilisation}
 \justifying{
 Dans l'url d'un service caché on remplace .onion avec .tor2web.org. 
Le blog de Stéphane Borztmeyer \url{http://7j3ncmar4jm2r3e7.onion}
\url{http://7j3ncmar4jm2r3e7.tor2web.org}
}
\end{block}
\end{frame}

%----------------------------------------------------------------------------------------
\begin{frame}
\frametitle{Tor hidden service - Tor2Web 3/4}

\begin{block}{Attention}
 \justifying{
Tor2web préserve l'anonymat des éditeurs de contenu (vu qu'ils sont sur des services cachés de Tor), mais n'est pas lui-même un outil d'anonymat et ne propose pas de protection aux utilisateurs autre que HTTPS. 
\\~\\
Un avertissement concernant la vie privée et la sécurité est ajouté à l'en-tête de chaque page web, encourageant les lecteurs à utiliser le Tor Browser pour être anonyme lorsqu'ils consultent le contenu du Service caché.
}
\end{block}
\begin{center}
\includegraphics[scale=0.6]{./images/Tor2web_attention_Suite.jpg}
\end{center}
\end{frame}

%----------------------------------------------------------------------------------------
\begin{frame}
\frametitle{Tor hidden service - Tor2Web 4/4}

\begin{block}{Communauté autour de Tor2Web}
 \justifying{
Comme Tor, Tor2web fonctionne à l'aide serveurs maintenus par une communauté ouverte d'individus et d'organisations. Il est donc possible de soutenir financièrement le projet, de contribuer au code ou d'installer son propre proxy permettant aux utilisateurs de consulter des services cachés.
}
\end{block}
\end{frame}

%----------------------------------------------------------------------------------------
\begin{frame}
\begin{center}
\Huge{Comment se connecter via Tor ? }
\\~\\ \includegraphics[scale=0.4]{./images/logo_tor.jpg}
\end{center}
\end{frame}
%----------------------------------------------------------------------------------------
\begin{frame}
\frametitle{Comment se connecter via Tor?}

\begin{block}{Le Tor Browser}
\justifying{
Le Tor Browser est basé sur la version Extended Support de Firefox, auxquelles sont ajoutée les extensions préconfigurées permettant qu’au lancement du navigateur, celui-ci se connecte \emph{au réseau Tor}. 
\\~\\$\Rightarrow$ Ainsi, toute la navigation qui se fait via ce navigateur est faite au travers du réseau Tor. 
}
\end{block}
\end{frame}

%----------------------------------------------------------------------------------------
\begin{frame}
\frametitle{Télécharger le Tor Browser}
\justifying{
Toutes les versions (dans différentes langues, différents OS) sont disponibles sur le site du projet : 
\\ \url{https://www.torproject.org/}
\\ Rq : Il existe la possibilité de le recevoir par mail...
}
\begin{center}
\includegraphics[scale=0.5]{./images/tor2.jpg}
\end{center}

\end{frame}

%----------------------------------------------------------------------------------------
\begin{frame}
\frametitle{Vérifier le Tor Browser téléchargé}
Via les clefs GPG, cf. le tuto sur le site de Tor. \\ \url{https://www.torproject.org/docs/verifying-signatures.html}
\begin{center}
\includegraphics[scale=0.6]{./images/VerifyingSignatures.jpg}
\end{center}
\end{frame}

%----------------------------------------------------------------------------------------
\begin{frame}
\frametitle{Installer le Tor Browser}

\begin{block}{Un logiciel \emph{comme} les autres}
\justifying{
Le Tor Browser s'installe comme n'importe quel logiciel Windows, OS X. (Des tutoriaux existent si besoin).
\\~\\
Rq : sous Windows, le Tor Browser déclenche une alerte avec la suite Symantec (faux positif). 
\\~\\
Pour Ubuntu, GNU/Linux c'est un programme autonome/portable. On peut aussi également l'installer en compilant les sources.
}
\end{block}
\end{frame}

%----------------------------------------------------------------------------------------
\begin{frame}
\frametitle{Lancer le Tor Browser}
\begin{center}
\includegraphics[scale=0.3]{./images/tor_browser03.jpg}
\end{center}
\end{frame}

%----------------------------------------------------------------------------------------
\begin{frame}
\frametitle{Comment être sûr qu'on est bien connecté à Tor?}

\begin{center}
\url{https://check.torproject.org/}
\\~\\
\includegraphics[scale=0.4]{./images/congrats.jpg}
\end{center}
\end{frame}

%----------------------------------------------------------------------------------------
\begin{frame}
\frametitle{Les nouveautés de la version 4.5 1/2}

\begin{center}
\includegraphics[scale=0.6]{./images/onionmenu.jpg}
\end{center}

\begin{block}{Pour la vie privée}
\begin{itemize}
 \justifying{
\item Visualisation du circuit emprunté (désactivable) ;
\item Changement de circuit par onglets  ;
\item Cloisonnement des applications tierces à l'onglet ;
\item Moteur de recherche par défaut : Disconnect (qui fournit des résultats de recherche Google).
}
\end{itemize}
\end{block}
\end{frame}

%----------------------------------------------------------------------------------------
\begin{frame}
\frametitle{Les nouveautés de la version 4.5 2/2}
\begin{center}
\includegraphics[scale=0.6]{./images/securityslider}
\end{center}
\end{frame}

\begin{frame}
\frametitle{Les nouveautés de la version 4.5 2/2}
\begin{block}{Le curseur de sécurité}
\begin{itemize}
 \justifying{
\item Haut - JavaScript est désactivé sur tous les sites par défaut, certains types d’images sont désactivées.
\item Moyen-Haut - Tous les optimisations de performances JavaScript sont désactivés, certains police fonctionnalités de rendu sont désactivées, JavaScript est désactivé sur tous les non-sites HTTPS par défaut.
\item  Moyen-Bas - HTML5 audio et vidéo sont en mode click-to-play, quelques optimisations de performances JavaScript sont désactivés, les fichiers JAR à distance sont bloqués et quelques méthodes pour afficher des équations mathématiques sont désactivées.
\item  Faible (par défaut) - Toutes les fonctions du navigateur sont activés.
}
\end{itemize}
La compatibilité diminue et la sécurité augmente avec chaque niveau de sécurité.
\end{block}
\end{frame}

%----------------------------------------------------------------------------------------
\begin{frame}
\begin{center}
\Huge{Maintenir le Tor Browser \\ à jour ? }
\\~\\ \includegraphics[scale=0.4]{./images/logo_tor.jpg}
\end{center}
\end{frame}

%----------------------------------------------------------------------------------------
\begin{frame}
\frametitle{Vérifier et installer les mises à jour}

Régulièrement, le TorBrowser est mis à jour (corrections de bugs, de failles, ajout de fonctionnalités).
\begin{center}
\includegraphics[scale=0.6]{./images/onionmenu.jpg}
\end{center}

\begin{block}{Depuis un TorBrowser}
\begin{itemize}
 \justifying{
\item Cliquer sur "Vérifier les mises à jour"
}
\end{itemize}
$\Rightarrow$  La mise à jour se fait via Tor.
\end{block}
\end{frame}

%----------------------------------------------------------------------------------------
\begin{frame}
\frametitle{Tor Browser Launcher}
Pour avoir un Tor Browser toujours à jour, on peut installer le Tor Browser Launcher.
\begin{center}
\url{https://github.com/micahflee/torbrowser-launcher}
\\ \includegraphics[scale=0.3]{./images/tor_browser01.jpg}
\\ \includegraphics[scale=0.3]{./images/tor_browser02.jpg}
\end{center}
\end{frame}
%----------------------------------------------------------------------------------------
\begin{frame}
\frametitle{Tor Browser Launcher}

\begin{block}{Ce qu'apporte le Tor Browser launcher}
Il ajout un lanceur d’application "Tor Browser" dans le menu de votre environnement de bureau. Au lancement, sont alors gérés de façon automatique en tâche de fond :
\begin{itemize}
\justifying{
\item le téléchargement de la version la plus récente de TBB, dans votre langue et pour votre architecture ;
\item la mise à jour (tout en conservant vos signets et préférences) manuel ;
\item la vérification de la signature GnuPG du TBB (pour être sûr de l’intégrité des fichiers).
}
\end{itemize}
\end{block}
\end{frame}
%----------------------------------------------------------------------------------------
\begin{frame}
\begin{center}
\includegraphics[scale=0.3]{./images/tails_logo.jpg}
\\~\\
\url{https://tails.boom.org}
\end{center}
\end{frame}
%----------------------------------------------------------------------------------------
\begin{frame}
\frametitle{Utiliser Tor - Tails}

\begin{block}{Tails (The Amnesic Incognito Live System)}
\justifying{
C'est un système d'exploitation complet basé sur Linux et Debian, en live.
 \url{https://tails.boom.org}
}
\end{block}
\begin{center}
\includegraphics[scale=0.3]{./images/tails.jpg}

\end{center}
\end{frame}
%----------------------------------------------------------------------------------------
\begin{frame}
\begin{center}
\Huge{Vous voulez que Tor \\ marche vraiment ?}
\\~\\ \includegraphics[scale=0.4]{./images/logo_tor.jpg}
\end{center}
\end{frame}
%----------------------------------------------------------------------------------------
\begin{frame}
\frametitle{Vous voulez que Tor marche vraiment ?}
\justifying{
Vous devrez changer quelques-unes de vos habitudes, et certaines choses ne marcheront pas exactement comme vous le voudrez.
}
\begin{block}{Ce qu'il faut faire et ne pas faire}
\begin{itemize}
\justifying{
\item Ne faîte pas de Torrent via Tor.
\item N'activez pas et n'installez pas de plugins dans le navigateur.
\item Utilisez dès que possible la version HTTPS des sites webs.
\item Comme avec un navigateur "normal", ne consultez pas/n'ouvrez pas de document téléchargé pendant que vous êtes connecté si ceux-ci présentent \emph{"un risque potentiel"}. 
\item Un noeud de sortie malveillant peut corrompre un binaire...
}
\end{itemize}
\end{block}
$\Rightarrow$ D'une façon générale, renseignez-vous, documentez vous sur Tor.
\end{frame}

%----------------------------------------------------------------------------------------
\begin{frame}
\begin{center}
\Huge{Limites à l'usage de Tor}
\\~\\ \includegraphics[scale=0.4]{./images/logo_tor.jpg}
\end{center}
\end{frame}

%----------------------------------------------------------------------------------------
\begin{frame}
\frametitle{Limites à l'usage de Tor}

\begin{block}{Utiliser Tor amène certaines contraintes pour l'utilisateur}
\begin{itemize}
\justifying{
\item Il n'y a pas de plugin flash (mais les vidéos HTML5 passe).
\item Il faut régulièrement activer le javascript (avec parcimonie).
\item Beaucoup de noeuds de sorties sont bloqués (Cloudfare) etc.
\item Il y a nécessité de saisir des captchas pour ne pas être assimilé à un bot (et d'activer le javascript).
\item On ne peut pas créer de compte (Gmail, Twitter...)
\item On sait qu'on utilise TOR (sauf si on utilise l'obfuscation).
}
\end{itemize}
\end{block}
\end{frame}

%----------------------------------------------------------------------------------------
\begin{frame}
\begin{center}
\Huge{Soutenir le projet Tor}
\\~\\ \includegraphics[scale=0.4]{./images/logo_tor.jpg}
\end{center}
\end{frame}
%---------------------------------------------------------------------------------------
\begin{frame}
\frametitle{Soutenir le projet Tor 1/3}

\begin{block}{NosOignons}
\justifying{
L'association NosOignons.net propose des nœuds de sortie Tor financés par la communauté.
\url{https://nos-oignons.net}
}
\begin{center}
\includegraphics[scale=0.6]{./images/Dons_nos_oignons.jpg}
\end{center}
\end{block}
 \end{frame}
%---------------------------------------------------------------------------------------
\begin{frame}
\frametitle{Soutenir le projet Tor 2/3}
\begin{block}{Tor Project}
\begin{itemize}
\justifying{
\item Devenir membre de la communauté Tor, Tails
\item Contibuez au code...
\item Faire des tutoriaux, de la traduction...
}
\end{itemize}
$\Rightarrow$  \url{https://www.torproject.org/}
\end{block}
 \end{frame}

%----------------------------------------------------------------------------------------
\begin{frame}
\frametitle{Soutenir le projet Tor 3/3}

\begin{block}{Enlever les noeuds de sortie Tor des listes noires}
\justifying{
Si vous utilisez Cloudflare pour protéger votre site, un script ajoute les relais/noeud de sortie sur une white-liste : 
\\ \url{https://github.com/DonnchaC/cloudflare-tor-whitelister}
\\~\\$\Rightarrow$ Cela permet aux utilisateur de Tor ne pas avoir à saisir de \emph{captcha}.
}
\end{block}
\end{frame}

%----------------------------------------------------------------------------------------
\begin{frame}
\begin{center}
\Huge{Questions et discussion}
\end{center}
\end{frame}
%----------------------------------------------------------------------------------------
\end{document}
