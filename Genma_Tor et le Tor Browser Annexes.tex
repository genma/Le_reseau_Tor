\documentclass{beamer}
\mode<presentation> {
\usepackage{color}
\definecolor{bottomcolour}{rgb}{0.21,0.11,0.21}
\definecolor{middlecolour}{rgb}{0.21,0.11,0.21}
\setbeamercolor{structure}{fg=white}
\setbeamertemplate{frametitle}[default]%[center]
\setbeamercolor{normal text}{bg=black, fg=white}
\setbeamertemplate{background canvas}[vertical shading]
[bottom=bottomcolour, middle=middlecolour, top=black]
\setbeamertemplate{items}[circle]
\setbeamertemplate{navigation symbols}{} %no nav symbols
\setbeamercolor{block title}{use=structure,fg=white,bg=structure.fg!50!red!50!blue!100!green}
\setbeamercolor{block body}{parent=normal text,use=block title,bg=block title.bg!5!white!10!bg,fg=white}
\setbeamertemplate{navigation symbols}{}
%-- Ajout du nb de pages en bas de page %
\addtobeamertemplate{footline}{\hfill\insertframenumber/\inserttotalframenumber\hspace{2em}\null}
}
\usepackage{graphicx} 
\usepackage{booktabs} 
\usepackage[utf8]{inputenc}  
\usepackage[T1]{fontenc}  
\usepackage{geometry}     
\usepackage[francais]{babel} 
\usepackage{eurosym}
\usepackage{verbatim}
\usepackage{ragged2e}
\justifying
\input{cc_beamer}
\title[Tor et le Tor Browser Bundle]{Le réseau Tor - Annexes} 
\author{Genma}
\date{SCE2015 - 13 juin 2015}
\begin{document}
\begin{frame}
	\titlepage
	\begin{center}
		\includegraphics[scale=0.2]{./images/logo_tor.jpg}
		\\		
		\CcGroupByNcSa{0.83}{0.95ex}\\[2.5ex]
		{\tiny\CcNote{\CcLongnameByNcSa}}
		\vspace*{-2.5ex}
	\end{center}
\end{frame}
%----------------------------------------------------------------------------------------
\begin{frame}
\begin{center}
\Huge{Annexes}
\end{center}
\end{frame}
%----------------------------------------------------------------------------------------

%----------------------------------------------------------------------------------------
\begin{frame}
\frametitle{Annexes}
\begin{itemize}
\justifying{
\item Polémiques
\item Quelques chiffres
\item Installer des extensions dans le navigateur Tor
\item Les types de noeuds
\item Vérification des signatures
\item La commande Torify
\item Torbirdy
\item Les bridges et l'obfuscation
\item Précisions sur le DNS
\item Les échanges de clefs
\item Le financement
\item Série de liens
}
\end{itemize}
\end{frame}

%----------------------------------------------------------------------------------------
\begin{frame}
\frametitle{Les polémiques}
\justifying{
\begin{block}{Utilisation pour des actes malveillants - Le "Darknet"}
Les sites Web des services cachés ne représentent que 2\% du trafic total dixit Roger Dingledine.\\
\url{http://www.theguardian.com/technology/2014/dec/31/dark-web-traffic-child-abuse-sites}
\end{block}
}

\justifying{
\begin{block}{Who use Tor? Tor user}
\url{https://www.torproject.org/about/torusers.html.en}
\end{block}
}
\end{frame}

%----------------------------------------------------------------------------------------
\begin{frame}
\frametitle{Quelques chiffres datant de nov 2013}
\begin{itemize}
\justifying{
\item 4.600 relais présents dans le réseau  ;
\item 900 noeuds de sortie ;
\item 40 Gb/s de bande passante ;
\item 25 Gb/s utilisés ;
\item 500k utilisateurs (6\% de français).
}
\end{itemize}
\end{frame}

%----------------------------------------------------------------------------------------
\begin{frame}
\frametitle{Installer des extensions dans le navigateur Tor?}

\justifying{
L'installation des add-ons dans le navigateur Tor peut ajouter des failles de sécurité potentielles :
}
\begin{itemize}
\justifying{
\item Certaines extensions pourraient ne pas prendre en compte la configuration (tout faire passer par Tor) et laisser fuiter des informations privées.
\item Certaines extensions pourraient  révéler des informations sur vos habitudes de navigation, l'historique de navigation, ou des informations du système, que ce soit à dessein ou par erreur.
\item Les extensions peuvent avoir des bogues et des failles de sécurité qui peuvent être exploitées à distance par un attaquant.
\item  Les extensions peuvent avoir des bogues casser la sécurité offerte par d'autres add-ons, par exemple Torbutton, et de casser votre anonymat.
\item Elles changent votre navigateur en le démarquant, rendant unique.
}
\end{itemize}
\justifying{
Sauf preuve du contraire, aucun add-on, en dehors de ceux déjà inclus dans le TorBrowser (qui ont été sérieusement vérifiés et peuvent être considérées comme sûr) ne sont à ajouter.
}
\end{frame}

%----------------------------------------------------------------------------------------
\begin{frame}
\frametitle{Les types de noeuds}
\begin{itemize}
\justifying{
\item Guards : noeuds d'entrée publics ;
\item Bridges : noeuds d'entrée publics ;
\item Midlle : noeuds intermédiaires ;
\item Exit : noeuds de sortie ;
\item Obfsproxy : noeuds d'obfuscation.
}
\end{itemize}
\end{frame}

%----------------------------------------------------------------------------------------
\begin{frame}
\frametitle{Vérification des signatures}
\justifying{
\begin{block}{Via la commande}
gpg --verify torbrowser-install-4.5.1exe.asc  torbrowser-install-4.5.1exe
\\~\\
Cf. \url{https://www.torproject.org/docs/verifying-signatures.html.en}
\end{block}
}
\end{frame}
%----------------------------------------------------------------------------------------
\begin{frame}
\frametitle{La commande Torify}
\justifying{
Torify est une commande qui, placée devant le nom d’une commande/d’un programme qui utilise le réseau, permet que ce dernier/cette dernière fasse passer son trafic par TOR. Ainsi, n’importe quelle application pourra passer par TOR au lieu de se connecter directement à Internet, et ce, à la demande de l’utilisateur.
}
\end{frame}
%----------------------------------------------------------------------------------------
\begin{frame}
\frametitle{TorBirdy}
\justifying{
TorBirdy est une extension pour le courielleur Thunderbird qui permet de faire la réception et l'envoi de mail en passant par Tor (on n'expose alors pas sa propre IP aux serveurs IMAP/SMTP).
}
\end{frame}

%----------------------------------------------------------------------------------------
\begin{frame}
\frametitle{TorBirdy}
\begin{center}
\includegraphics[scale=0.45]{./images/torbirdy01.jpg}
\end{center}
\end{frame}
\begin{frame}
\frametitle{TorBirdy}
\begin{center}
\includegraphics[scale=0.45]{./images/torbirdy02.jpg}
\end{center}
\end{frame}
\begin{frame}
\frametitle{TorBirdy}
\begin{center}
\includegraphics[scale=0.45]{./images/torbirdy03.jpg}
\end{center}
\end{frame}
\begin{frame}
\frametitle{TorBirdy}
\begin{center}
\includegraphics[scale=0.45]{./images/torbirdy04.jpg}
\end{center}
\end{frame}


%----------------------------------------------------------------------------------------
\begin{frame}
\frametitle{Les bridges et l'obfuscation}
\justifying{
Les Bridges sont des relais Tor qui ne sont pas listés dans l'annuaire principal de Tor. 
\\Un bridge obfsproxy permet d'obfusquer le trafic Tor, c'est à dire cacher les connexions au réseaux Tor (encapsulation dans du trafic neutre).
}

\begin{center}
\includegraphics[scale=0.45]{./images/tor_bridge.jpg}
\end{center}
\end{frame}

%----------------------------------------------------------------------------------------
\begin{frame}
\frametitle{Précisions sur le DNS}
\justifying{
Tor ne peut assurer la protection de paquets UDP, et n’en soutient donc pas les utilisations, notamment les requêtes aux serveurs DNS. 
\\~\\
Cependant Tor offre la possibilité d'acheminer les requêtes DNS à travers son réseau, notamment à l’aide de la commande \emph{torsocks}.
}
\end{frame}


%----------------------------------------------------------------------------------------
\begin{frame}
\frametitle{Les échanges de clefs}
\justifying{
Les paquets à acheminer sont associés à une identification du propriétaire du circuit (la personne qui l’a construit). Mais cette identification est un code arbitraire qui a été choisi au moment de la construction du circuit. L’identification réelle du propriétaire est inaccessible.
\\~\\
Cette construction fait appel au concept de cryptographie hybride. Chaque nœud d’oignon possède une clef publique, mais la cryptographie à clef secrète (clef symétrique) est bien plus rapide que celle à clef publique. L’idée est donc de distribuer à chaque nœud du circuit une clef secrète chiffrée avec leur clef publique.
\\~\\
Après la phase de construction, chaque nœud du circuit dispose d'une clef secrète qui lui est propre et ne connaît que son prédécesseur et son successeur au sein du circuit.
\\
Source \url{https://fr.wikipedia.org/wiki/Tor_(réseau)}
}

\end{frame}

%----------------------------------------------------------------------------------------
\begin{frame}
\frametitle{Le financement}
Le projet coûte 2 M\$ annuellement pour son développement et pour payer les nombreux serveurs. En 2012 :
\begin{itemize}
\justifying{
\item 60\% proviennent du gouvernement américain (soutien à la liberté d'expression et à la recherche scientifique) ;
\item 18\% proviennent de fondations et autres donateurs (John S. and James L. Knight Foundation (en), SRI International, Google, Swedish International Development Cooperation Agency;
\item 18\% proviennent de la valorisation des contributions des bénévoles.
}
\end{itemize}
Source \url{https://fr.wikipedia.org/wiki/Tor_(réseau)}
\\
\justifying{
Ce que financent plusieurs branches différentes du gouvernement des USA est le développement des logiciels à travers l'organisation The Tor Project. Le réseau Tor est quant à lui mis en place par des bénévoles et des organisations comme Nos oignons. 
}
\end{frame}

%----------------------------------------------------------------------------------------
\begin{frame}
\frametitle{S'informer - Série de liens}
\begin{itemize}
\justifying{
\item \url{https://www.torproject.org}
\item \url{https://blog.torproject.org}
\item \url{https://tails.boum.org/}
}
\end{itemize}
\end{frame}

%----------------------------------------------------------------------------------------
\end{document}
